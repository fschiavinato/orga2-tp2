% ******************************************************** %
%              TEMPLATE DE INFORME ORGA2 v0.1              %
% ******************************************************** %
% ******************************************************** %
%                                                          %
% ALGUNOS PAQUETES REQUERIDOS (EN UBUNTU):                 %
% ========================================
%                                                          %
% texlive-latex-base                                       %
% texlive-latex-recommended                                %
% texlive-fonts-recommended                                %
% texlive-latex-extra?                                     %
% texlive-lang-spanish (en ubuntu 13.10)                   %
% ******************************************************** %


\documentclass[a4paper]{article}
\usepackage[spanish]{babel}
\usepackage[utf8]{inputenc}
\usepackage{charter}   % tipografia
\usepackage{graphicx}
\usepackage{makeidx}
\usepackage{paralist} %itemize inline

\usepackage{float}
\usepackage{amsmath, amsthm, amssymb}
\usepackage{amsfonts}
\usepackage{sectsty}
\usepackage{charter}
\usepackage{wrapfig}
\usepackage{listings}
\lstset{language=C}

 \setcounter{secnumdepth}{2}
\usepackage{underscore}
\usepackage{caratula}
\usepackage{url}


% ********************************************************* %
% ~~~~~~~~              Code snippets             ~~~~~~~~~ %
% ********************************************************* %

\usepackage{color} % para snipets de codigo coloreados
\usepackage{fancybox}  % para el sbox de los snipets de codigo

\definecolor{litegrey}{gray}{0.94}

\newenvironment{codesnippet}{%
	\begin{Sbox}\begin{minipage}{\textwidth}\sffamily\small}%
	{\end{minipage}\end{Sbox}%
		\begin{center}%
		\vspace{-0.4cm}\colorbox{litegrey}{\TheSbox}\end{center}\vspace{0.3cm}}



% ********************************************************* %
% ~~~~~~~~         Formato de las páginas         ~~~~~~~~~ %
% ********************************************************* %

\usepackage{fancyhdr}
\pagestyle{fancy}

%\renewcommand{\chaptermark}[1]{\markboth{#1}{}}
\renewcommand{\sectionmark}[1]{\markright{\thesection\ - #1}}

\fancyhf{}

\fancyhead[LO]{Sección \rightmark} % \thesection\ 
\fancyfoot[LO]{\small{Nombre Apellido, Nombre Apellido, Nombre Apellido}}
\fancyfoot[RO]{\thepage}
\renewcommand{\headrulewidth}{0.5pt}
\renewcommand{\footrulewidth}{0.5pt}
\setlength{\hoffset}{-0.8in}
\setlength{\textwidth}{16cm}
%\setlength{\hoffset}{-1.1cm}
%\setlength{\textwidth}{16cm}
\setlength{\headsep}{0.5cm}
\setlength{\textheight}{25cm}
\setlength{\voffset}{-0.7in}
\setlength{\headwidth}{\textwidth}
\setlength{\headheight}{13.1pt}

\renewcommand{\baselinestretch}{1.1}  % line spacing

% ******************************************************** %


\begin{document}


\thispagestyle{empty}
\materia{Organización del Computador II}
\submateria{Segundo Cuatrimestre de 2014}
\titulo{Trabajo Práctico II}
\subtitulo{subtitulo del trabajo}
\integrante{Nombre}{XXX/XX}{mail}
\integrante{Nombre}{XXX/XX}{mail}

%\maketitle
\newpage

\thispagestyle{empty}
\vfill
\begin{abstract}
En el presente trabajo se describe la problemática de ...
\end{abstract}

\thispagestyle{empty}
\vspace{3cm}
\tableofcontents
\newpage


%\normalsize
\newpage

\section{Objetivos generales}
\paragraph{}
El objetivo de este Trabajo Práctico es el de, mediante la implementación y comparación de distintos códigos de filtros de procesamiento de imágenes,  el de   ejercitar y demostrar nuestro entendimiento de las SIMD del assembler, así como la capacidad de hacer un análisis comparativo entre las versiones de assembler y c, de sus ventajas y las limitaciones de cada  código, así como las restricciones del filtro en si, de una manera clara y demostrativa. 
%\section{Contexto}
%\begin{figure}
%  \begin{center}
        %\includegraphics[scale=0.66]{}
%       \caption{Descripcion de la figura}
%       \label{nombreparareferenciar}
        
        
%       hola como va
%  \end{center}
%\end{figure}
%\paragraph{\textbf{Titulo del parrafo} } Empezando
%\ref{nombreparareferenciar}.
%\begin{codesnippet}
%\begin{verbatim}
%struct Pepe {};
%\end{verbatim}
%\end{codesnippet}
\section{Enunciado y solución} 
%\input{enunciado}
\section{Descripciones}
\subsection{Cropflip}
\paragraph{\textbf{Aridad}}
En el filtro cropflip se nos pasan los siguientes parámetros:
\hfill \break
\\
- tamx: Cantidad de píxeles de ancho del recuadro a copiar.
\hfill \break
- tamy: Cantidad de píxeles de alto del recuadro a copiar.
\hfill \break
- offsetx: Cantidad de píxeles de ancho a saltear de la imagen fuente.
\hfill \break
- offsety: Cantidad de píxeles de alto a saltear de la imagen fuente.
\hfill \break
Se asumen que todos son múltiplos de 16, y se toman estos dos parámetros
\hfill \break
\\
-scr_row_size: tamaño de una fila (en bytes) de la imagen de entrada
\hfill \break
-dst_row_size: tamaño de una fila (en bytes) de la imagen de salida
\hfill \break
asumiendo a su vez que nos pasan los punteros a una imagen de entrada y de salida. Cualquier tipo de shifteo, si pongo un inmediato para indicar la cantidad, son bits
\hfill \break
\paragraph{\textbf{Descripcion de cropflip.c}}
\hfill \break
        En la implementación de cropflip_c.c primero declaramos todas las variables que vamos a utilizar, y luego  utilizamos un for anidado en otro para recorrer la imagen fuente y copiarla modificada a la imagen de destino.
\hfill \break
        Mas claramente, tenemos dos contadores j e i, representantes de las dimensiones del recorte que le hacemos a la imagen. j cuenta el numero de columnas e i el numero de filas. El primer for del ciclo incrementa las filas, mientras que el anidado incrementa las columnas, incrementando los contadores desde 0 hasta llegar a sus representaciones, realizando así las acciones del filtro en toda la imagen.
\hfill \break   
         En cuanto a las acciones del filtro, píxel x píxel, dentro del segundo for, utilizamos dos punteros a los pixels actuales de la imagen de entrada (source) y la  de destino, llamados p_s y p_d.
\hfill \break 
         El primero es direccionado a la posición de i y j en cada ciclo, recorriendo la matriz píxel por píxel en el orden de las direcciones. (se multiplica por 4 j porque un píxel tiene 4 bytes) El segundo, por cada entrada del ciclo, hace el sgte calculo. En el lado del las filas se va a la posición que indica es la de inicio (marcada por un integer llamado offsety), se le suma el tamaño de la imagen offset que queremos (guardada en tamx) y se le resta por cada ciclo el i de la fila actual(restándole 1 por el 0), quedando así apuntando a la fila inversa a la que estamos en la de destino, como queremos hacer. Asimismo, en cada fila, nos colocamos en la columna necesaria sumándole a la posición inicial dada en tamx(análogo a tamy) el valor actual de j (\textit{ $*4$ , para avanzar de a píxel}).
         \hfill \break
Allí, en cada ciclo el p_s esta apuntando la posición donde esta el píxel que debería estar en la p_d, por lo que lo único que hay que hacer es igualar sus punteros y seguir el ciclo, y al terminar se tiene cambiada toda la imagen.
\hfill \break
\paragraph{\textbf{Descripcion de cropflip.asm}}
\hfill \break
        
        Los parámetros de cropflip_asm.asm están pasados de la siguiente manera:
        \hfill \break
         rdi = src
         \hfill \break
     rsi = dst
     \hfill \break 
     edx = cols
     \hfill \break
     ecx = filas
     \hfill \break
     r8d = src_row_size
     \hfill \break
     r9d = dst_row_size
     \hfill \break
     [rsp+8] = tamx
     \hfill \break
     [rsp+16] = tamy
     \hfill \break
     [rsp+24] = offsetx
     \hfill \break
     [rsp+32] = offsety
     \hfill \break
  *extracto de asm
  \hfill \break
     
     siendo src y dst los punteros a las imágenes de entrada y salida.
    \hfill \break
    \\
     En la función, luego de hacer la pila y vaciar los registros (cosa que, salvo el r12 y r14, hacemos por mera precausion) movemos a la parte baja de los registros los datos que nos quedaron en la pila para evitarnos múltiples accesos de memoria en el ciclo, y luego multiplicamos r12 (que tiene el tamx) *4 para pasar de cantidad de píxeles a bytes. (cada píxel tiene 4 bytes)
     \hfill \break
     Luego, se vacía rax y se le suma r15d (donde quedo guardado offsety) y r13d (donde quedo guardado tamy ) - 1.
     Hasta ahora rax = offsety + tamy - 1. (parecido a la posición de filas de la versión c) luego, lo multiplicamos por r8d (que es el tamaño de fila), al hacerlo, ahora rax contiene el valor necesario para que un puntero, al sumárselo, avance offsety + tamy - 1 filas. Naturalmente, le sumamos esto a rdi, que tiene el puntero a la dirección inicial de la imagen de entrada, para moverla a la fila que queremos. luego, direccionamos rdi, a la columna pedida usando r14 (offsetx) x 4 (bytes) y un lea. Posicionados donde debe iniciar el ciclo (notar que estamos en la misma posición en que un ciclo de la versión c estaría  en la primera ejecución);
     \hfill \break
      Entramos al ciclo. (vamos a usar vaciado rax y rbx como contadores/punteros)
       Este es muy sencillo. Usamos a xmm0 para mover 4 píxeles, desde contenido de rdi + rbx al contenido de rci + rbx.
      a rbx,  puntero a columna actual, le sumamos lo necesario para ir al siguiente píxel a procesar y lo comparamos con la cantidad de columnas que nos piden, si es menor, seguimos en la misma fila, por lo que volvemos al ciclo.       
      Si no es el caso, entonces debemos pasar a la fila siguiente. Vaciamos rbx para que el contador vuelva a la columna 0 (podemos hacer esto porque los tamaños son múltiplos de 4); reducimos rdi x una fila (r8d) y le agregamos a rsi una (en r9d, son iguales). Notar como estas orden son paralelas al incremento de i  en la implementación c (en el direccionamiento).
      \hfill \break      
      Incrementamos rax para contar que pasamos una fila, y lo comparamos con la cantidad de filas a pasar, si es menor, volvemos al ciclo, si no lo es, toda la imagen esta pasada, por lo que desarmamos la pila y salimos.
      \hfill \break
      En esta versión, debemos mover el puntero de la imagen de entrada, pero hacemos este direccionamiento una vez y no muchas, ya que luego vamos restando. También, como procesamos varios píxeles a la vez, tarda menos ciclos.
             
     
          
     
     
     
     
      
       
      
 
          
         
\subsection{Sepia}
\paragraph{\textbf{Aridad}}
\hfill \break
En el filtro sepia definimos los  siguientes parámetros (que vienen implícitos con la imagen de entrada, o nosotros definimos ):
\hfill \break
\\
src = puntero a dirección de píxel inicial de la imagen de entrada
\hfill \break
dst = puntero a dirección de píxel inicial de la imagen de salida
\hfill \break
cols = cantidad de columnas de la imagen
fils = cantidad de filas de la imagen
src_row_size = tamaño de fila de la imagen de entrada
dst_row_size = tamaño de fila de la imagen de salida 
Se asumen que todos son múltiplos de 16.Cualquier tipo de shifteo, si pongo un inmediato para indicar la cantidad, son bits
\paragraph{\textbf{Descripcion de sepia.c}}
\hfill \break
        En la implementación de sepia_c.c tenemos de nuevo un doble ciclo anidado, usando a j e i como contadores de nuevo, representado lo mismo que en cropflip_c.c. Lo mismo ocurre con p_d y s_d. El recorrido de los for es el mismo, también, siendo lo único que cambia que esta delimitado por filas en vez de tamy y cols en vez de tamx (lo cual tiene lógica porque estamos quieren ver la imagen fuente completa, no solo una parte) y  lo que ocurre en el for anidado por cada iteración del ciclo hasta la salida de este, por lo tanto, focalizaremos en eso.
        \hfill \break
        En este filtro, a diferencia del cropflip, cada píxel del de entrada se corresponde con el de salida, por lo que tanto para p_d como para s_d se direcciona a la fila i de la columna j.(se multiplica por 4 j porque un píxel tiene 4 bytes)
        \hfill \break
        Luego, por lo que se hace una vez posicionado, básicamente es la formula del sepia. Se utiliza un short suma, al cual se le suman los valores de rojo, verde y azul. y luego a cada parte del píxel actual de la imagen de salida se le asigna esa suma multiplicada por se consiguiente numero. (siendo, 0.2 para azul, 0.3 para verde, 0.5 para rojo, y manteniendo igual el a)
        \hfill \break
        En cuanto al rojo, para calcular el valor que va destinado a ese lugar se shiftea la suma a la izq (dividiendo x 2), y, como es el único valor donde pudo haber quedado overflow, se le pregunta si no es mayor al valor máximo de unsigned de byte, si lo es, se reemplaza al valor máximo. Se satura el valor.
        \hfill \break
        
         
\paragraph{\textbf{Descripcion de sepia.asm}}
\hfill \break
        En la implementación de sepia_asm.asm nos pasan los sgtes parámetros en los sgtes registros
\hfill \break
        rdi = src
        \hfill \break
    rsi = dst
    \hfill \break
    edx = cols
    \hfill \break
    ecx = filas
    \hfill \break
    r8d = src_row_size
    \hfill \break    
    r9d = dst_row_size
    \hfill \break
 *extracto de sepia_asm.asm
  \hfill \break
  Primero armamos la pila y como en el cropflip_asm.asm multiplicamos  *4 (con shift) las columnas para que sea el numero de bytes x columna, en vez de pixels.
  Luego, vaciamos rax y rbx para usarlos de contador/punteros, y vaciamos la parte alta de los registros r8 y r9, que luego usaremos. 
  \hfill \break
   En el ciclo movemos xmm0 a la dirección del puntero a la imagen de entrada (rdi) + el contador de columnas, y agarramos, 4 píxeles. Luego, copiamos el registro en xmm1 y xmm9 con movdqa (alineado por asunciones)
   \hfill \break
   Luego, aplicamos punpckhbw con xmm8 (que, antes de entrar al ciclo, pusimos en 0) que pone en el registro de destino los bytes con mas significativos de ambos registros intercalados, siendo el mas alto el mas alto del registro de entrada (source). En este caso, xmm8 es la entrada, por lo cual el efecto es el de extender los dos primeros datos de bytes a words, contenidos en xmm0. Usamos, una instrucción análoga que se refiere a la parte baja para hacer lo mismo con xmm1, ahora teniendo los 4 píxeles en dos registros, con cada elemento siendo de tamaño word.
   \hfill \break
   Copiamos el contenido del registro en  xmm0 en xmm2 y el de xmm1 en xmm3 y los shifteamos a la derecha(de a quadruple, osea, de a mitades de registro) 8 lugares,para poner los a en 0 (cave resaltar que cambiamos el orden de rojo, verde y azul, también).
   \hfill \break
   Luego sumamos ambos registros horizontalmente con la instrucción phaddw. Esta, como su nombre lo indica, ve al registro destino como un lugar para 8 words, y rellena las cuatro posiciones mas significativas de este agrupando los datos que había en ella, de dos en dos, siendo la posición (word)  mas significativa nueva la suma de las dos words mas altas de el destino antes de la suma, y sucesivamente. Para las cuatro posiciones menos significativas del registro destina,  realizaba el mismo agrupamiento pero sumando las words del operando fuente en lugar del de destino. Nosotros realizamos estas suma entre xmm2 y xmm3, y luego del registro destino entre estos (xmm2) consigo mismo para conseguir la suma de elementos del píxel. seria un poco así:
   \hfill \break
   
   Registros antes de la suma horizontal (agrupados x words)
   \hfill \break
  $ xmm2 :| r0 | g0 | b0 | 0 | r1 | g1 | b1 | 0 | $
  $ xmm3 :| r2 | g2 | b2 | 0 | r3 | g3 | b3 | 0 |$
   \hfill \break
   luego de la primera suma horizontal (solo nos importa el destino)
   \hfill \break
   $xmm2 = | r2+g2 | b2 | r3+g3 | b3 | r0+g0 | b0 | r1+g1 | b1 |$
   \hfill \break
   luego de la segunda suma horizontal 
   \hfill \break
   $xmm2 = | r2+g2+b2 | r3+g3+b3 | r0+g0+b0 | r1+g1+b1 | r2+g2+b2 | r3+g3+b3 | r0+g0+b0 | r1+g1+b1 |$
   \hfill \break
   \\
   La notación usada separa cada xmm en 8 posiciones de word, en donde cada una anota el contenido de la siguiente manera... a,b,r u g representa el elemento de un píxel (r es rojo, b azul, g verde, y a el de alineación). A su vez, el numero que la acompaña  representa cual de los píxeles de los 4 que proceso es el que esta guardado en esa posición y le corresponde ese elemento.
   \hfill \break
   \\
   
   Ahora en xmm2 tenemos la suma de elementos para 4 píxeles (repetida dos veces) lo copiamos en xmm4 y lo shifteamos x word a la izquierda (/2 cada palabra), astutamente así obtenemos los nuevos componentes rojo de cuatro píxeles\hfill \break
   Ahora usamos la instrucción punpckhwd entre smm2 y xmm8 para hace algo parecido a lo de antes,  pasar los datos en xmm2 de word a doubleword(perdiendo los 4 menos significativos, pero como se repiten no nos importa) luego usamos la instrucción cvtdq2ps para convertir los datos a floats (porque vamos a mutiplicar con floats), lo copiamos a xmm3 y multiplicamos, usando mulps, xmm2 con xmm5 y xmm3 con xmm6 (donde se guardan coefb y coefg, respectivamente) que multiplica floats de precisión simple. Ahora tenemos en xmm2 y xmm3 los valores de azul y verde nuevos de los cuarto píxeles procesados. Usamos cvttps2dq para volver a doublewords enteras (que son mas rápidas)\hfill \break
   Ahora solo nos queda agrupar los píxeles de  manera correspondiente., usamos packusdw entre xmm2 y xmm3 para, ahora que no tenemos que hacer mas operaciones, convertirlas en words (el objetivo aquí es volver a tener los 4 píxeles en un registro) de manera saturada.(la parte alta de cada doubleword debería estar en 0, así que no hay problema) packusdw hace eso, en las 4 posiciones mas significativas los de source achicados, en los otras 4  las del destino.\hfill \break
   \\
   Luego, hacemos packuswb, que hace lo mismo que la instrucción anterior pero de words a bytes, entre xmm2 y xmm4. Al final el registro resultante es este:
   \hfill \break \\
    xmm2 = $ | r2' | r3' | r0' | r1' | r2' | r3' | r0' | r1' | g2' | g3' | g0' | g1' | b2' | b3' | b0' | b1' | $ \hfill \break
    \\
    (ahora veo las posiciones del registro de a bytes, por lo que hay 16 posiciones, empezando del 0)
    \\
    \hfill \break
    En este registro, aunque logramos juntar todos los componentes que queríamos (excepto las a) tenemos el problema de que las cosas están rotadas (por lo de antes con el a, ) y no están los pixels en orden.
    \hfill \break
        Para arreglarlo, usamos pshufb entre xmm2 y xmm7. (en xmm7 esta guardada una mascara que definimos en section,data). Pshufb intercambia los bytes del registro de destino de acuerdo a las posiciones instruccionadas en el registro fuente para cada uno. (si pongo ff se pone 0)
        \hfill \break
        En este caso:
        \hfill \break
        
    Lo que tengo en xmm2 (arriba la posición de registro, abajo el contenido)
    \hfill \break
 $|  f  |  e  |  d  |  c  |  b  |  a  |  9  |  8  |  7  |  6  |  5  |  4  |  3  |  2  |  1  |  0  |$
 \hfill \break
$ | r2' | r3' | r0' | r1' | r2' | r3' | r0' | r1' | g2' | g3' | g0' | g1' | b2' | b3' | b0' | b1' | $
 \hfill \break
 
 Lo que quiero en xmm2  
 \hfill \break
 $|  f  |  e  |  d  |  c  |  b  |  a  |  9  |  8  |  7  |  6  |  5  |  4  |  3  |  2  |  1  |  0  |$
 \hfill \break
$ |  0  | r0' | g0' | b0' |  0  | r1' | g1' | b1' |  0  | r2' | g2' | b2' |  0  | r3' | g3' | b3' |$
\hfill \break
La mascara en xmm7 (copiada antes del inicia del ciclo, definida en section.data), en hexagesimal.
\hfill \!!br0ken!!
$ |  f  |  e  |  d  |  c  |  b  |  a  |  9  |  8  |  7  |  6  |  5  |  4  |  3  |  2  |  1  |  0  |$
\hfill \break
 0x02, 0x06, 0x0a, 0xff, 0x03, 0x07, 0x0b, 0xff, 0x00, 0x04, 0x08, 0xff, 0x01, 0x05, 0x09, 0xff
 \hfill \break
 
 
De esta manera, gracias a esta instrucción, casi termino de reubicar los píxeles, ahora están en orden. Solo falta colocar las a. Para ello, agarro xmm9, 
( xmm9 : $| a0 | r0 | g0 | b0 | a1 | r1 | g1 | b1 | a2 | r2 | g2 | b2 | a3 | r3 | g3 | b3 |$) y le aplico  psrld, que shiftea de a doublewords  hacia la derecha, y  
 la cantidad de lugares a shiftear (debo llenar con 0 tres elementos, cada uno tiene 8 bytes, entonces multiplico por 3 por  eso).
 \hfill \break 
 Utilizo psrld inmediatamente después (con los mismos parámetros) para que las a del registro xmm9 quede en los lugares libres de el registro xmm2; entonces sumo de a byte estos dos (usando paddb) y muevo el resultado a la dirección de la imagen de destino correspondiente.
 \hfill \break
 
 Para terminar, sigo el ciclo análogamente a como lo hice en cropflip, con la excepción de que ahora tanto rdi como rsi avanzan a la fila siguiente (en vez de  a la anterior), y al terminar de recorrer la imagen desarmo la pila y salgo.
 \hfill \break
 
 
        
        
          
   
     
    
    
        
\subsection{LDR}
\paragraph{\textbf{Aridad}}
\hfill \break
En el filtro LDR se nos pasan los siguientes parámetros:
- alpha: Valor entre -255 y 255 que indica la intensidad del filtro.
Además, asumimos como parámetro de entrada implícitos:
src = puntero a dirección de píxel inicial de la imagen de entrada
\hfill \break
dst = puntero a dirección de píxel inicial de la imagen de salida
\hfill \break
cols = cantidad de columnas de la imagen
fils = cantidad de filas de la imagen
src_row_size = tamaño de fila de la imagen de entrada
dst_row_size = tamaño de fila de la imagen de salida 
Las filas, columnas pasadas son múltiplos de 16. Cualquier tipo de shifteo, si pongo un inmediato para indicar la cantidad, son bits
 \paragraph{\textbf{Aclaraciones}}
\hfill \break   
        En la implementación de ldr_c.c, para una lectura mas clara del código, definimos unas cuantas etiquetas que aclarare al momento y se usaran en el código. Estas son
        \hfill \break   
        MIN (x,y),  que dados dos parametros comparables nos devuelve el mínimo
        \hfill \break   
        MAX (x,y), la contraparte de mínimo
        \hfill \break   
        P =  2 y sirve para bordes.
        \hfill \break   
        MAXSUM = 4876875, una constante de la formula del filtro.
        \hfill \break   
        También, vamos a utilizar el termino, píxeles vecinos, para referirnos a los píxeles continuos al procesado (horizontal, vertical u diagonalmente) y  continuos a estos. (en un contorno de 3x3 al píxel actual en el filtro)   
        \hfill \break   
\paragraph{\textbf{Descripcion de LDR.c}}
\hfill \break    
         En la función en si tenemos dos ciclos anidados, con los mismos bordes y avances que el sepia, por lo cual no andaremos en detalles. 
        En cuanto a las acciones que realiza cada filtro en cada iteración, están descriptas a continuación:
        \hfill \break   
        
        Nos posicionamos en la fila i, en la columna j (*4, por la cantidad de bytes que ocupan los píxeles) de tanto el puntero a la imagen de entrada como el de salida (que definimos p_s y p_d, respectivamente) como este filtro no juega con las posiciones pero la distribución de la paleta, recorremos las dos imágenes paralelamente, avanzado de a píxel x iteración de ciclo, por toda una fila para el anidado, y por todas las filas en el ciclo completo.
        \hfill \break   
        Luego, debido a que los bordes de la imagen no deben ser modificados (contorno de 2 píxeles), usamos un if para comprobar si i o j representan las primeras columnas o filas, o las ultimas. En caso de que la guarda se cumple, simplemente designamos al puntero que apunte a la posición de entrada, aprovechándonos del aliasing para pasar intercambiado el valor a la imagen de salida. y continuamos el ciclo.
        \hfill \break   
        De no ser así, estamos en uno de los píxeles a los que debemos aplicarle la formula del filtro, Definimos 3 long long r, g y b (Necesitamos 64bits porque al hacer ultima la división el numero mas grande posible es 255*255+255*4876875=4A20DEB6), y hacemos un subciclo de for (con un for anidado) en el cual sumamos, sumamos en el r los componentes de color del píxel; el contenido de sus vecinos en un recuadro de 3x3, como mostraba la formula del enunciado. (utilizamos un puntero que va apuntado al píxel iterado, se mueve sumándole a p_s (fila actual en recorrido principal) el apunte a fila de vecino actual(l), y le sumamos k*cols para posicionarnos en la columna del vecino a procesar/ El p definido se usar  para delimitar los marcos del subciclo, en cada for del subciclo yendo desde -p hasta p, avanzando de a uno).
        \hfill \break   
        Ahora que tenemos   la suma en r, la multiplicamos por el alfa pasado, y los copiamos a b y g.(solo multiplicamos en 1 x que es mas rápido) \hfill \break       
        Realizamos la formula de componentes en r, g y b ( en facto, multiplicamos el contenido por el valor del componente del pixel dactual homologo, lo dividimos x MAXSUM y le sumamos a eso el valor del comoponente del pixel actual)
        \hfill \break   !!br0ken!!
        
        Con esto, ya tenemos en r g, b cada  valor que queríamos para cada componente del píxel, salvo que puede estar desfasado, por haber superado el max o mínimo permitido. Para evitar eso,  a cada componente del píxel de salida (al cual nos referimos con el puntero declarado anteriormente p_d) y colocamos en el, el mínimo entre 0 y el máximo de r, g o b y UCHAR_MAX(dado en limits.h), según corresponda, saturándolo tanto superior como inferiormente.
        \hfill \break   
        Seguimos el ciclo hasta que termine, cuando ya aplicamos el ciclo para toda la imagen, y salimos.
        \hfill \break   
        
        
\paragraph{\textbf{Descripcion de LDR.asm}}
\hfill \break           
        En la implementación de asm nos pasan los siguientes parámetros en los sgtes registros:
        \hfill \break
   rdi = src
   \hfill \break
   rsi = dst
   \hfill \break
   edx = cols
   \hfill \break
   ecx = filas
   \hfill \break
   r8d = src_row_size
   \hfill \break
   r9d = dst_row_size
   \hfill \break
   rbp+8 = alpha (la pila)
   \hfill \break
*extracto de ldr_asm.asm
\hfill \break
        También, al principio del programa, tenemos definidos varias constantes que vamos a utilizar en el programa. En la descripción diremos su nombre su nombre entre {} y entre paréntesis () el valor definido en ellos. (ej: {ejemplo(0)}.
        \hfill \break
         La excepción son las etiquetas max y max_f, definidas en las section. data del código en las cuales se definen 4 doublewords/floats de precisión simple consecutivas respectivamente, (esto representa la constante max que requiere la formula del filtro)
        En particular 
        \hfill \break
        max: 4876875, 4876875, 4876875,48676875
        \hfill \break
        max_f: 4876875.5, 4876875.5, 4876875.5, 4876875.5
        \hfill \break
        
        En cuanto a la función en si, primero armamos la pila. Luego, armamos los contadores, para lo cual  limpiamos la parte alta de los registros que vamos a usar (shifteando, para evitar ir a memoria) o los vaciamos con xors. Movemos a r10 el alpha para no acceder a memoria cada vez que lo necesitemos, y guardamos en r14d el numero de filas y en r15d el de columnas. Vamos a usarlos como representantes de los bordes, por lo cual le restamos a r14 2, para obtener el índice de fila en la cual tenemos que dejar que procesar, y 4 a r15, para obtener la columna en la cual debemos dejar de procesar.      
        (\textit{ desfasado a 2, o sea cuando rdi es cero procesamos los píxeles {(i,2),(i,3),(i,4),(i,5)}, siendo i la fila actual})
        \hfill \break 
        Utilizamos r11 como el índice de fila, y r12 como el de columna, (antes de entrar al ciclo xorceados a 0)  y movemos a xmm14 max y xmm15 max_f, para cuando los necesitamos no ir  a memoria
        \hfill \break
        \\
        
        El recorrido del ciclo en si es de la continua manera:
        \hfill \break
        \\
        Nos fijamos en cada iteración, si la dirección a la que apuntamos, pertenece a alguno de los bordes. Para ello, ponemos una etiqueta borde_izq, donde se compara al contador de columnas con (r12d) con 2. Si es menor  entonces es del borde y lo único que hace es mover la dirección apuntada por el puntero a la imagen de entrada (rdi) + la columna actual (r12 * 4 x lo de los píxeles) a r13, y este a la dirección apuntada por la dirección de salida (rsi) + la columna actual  (fijarse  que rdi y rsi tienen un aumento paralelo en toda la ejecución del programa). \hfill \break
Notar además que aunque estamos manejando píxeles estamos usando el registro r13, esto se debe a que solo dos pares de columnas(conjuntas) pertenecen al caso borde, por lo que si usáramos un xmm para hacer el cambio, agarraríamos mas píxeles de los cuales se constituye la excepción. Lo mismo ocurre en el borde_der, donde también los usamos.\hfill \break
        Sucesivamente, si  r12d fuera mayor o igual a 2, u omitiendo estos pasos, nos metemos en la etiqueta borde_inf, donde nos fijamos si el contador de filas (r11) es menor a 2. Si lo es , realizamos la misma transacción que en la etiqueta anterior entre imágenes de entrada y salida (pero usando xmm0 como intermediario, ya que en este borde se puede llevar de cuatro píxeles, y sin miedo al desfajase, por ser múltiplo de 16), si no lo es (o luego de hacerlo) entramos a la etiqueta borde_sup, donde comparamos que r11d sea menor a r14d (donde debíamos dejar de iterar filas)  y   si lo es pasamos a borde derecho. Si no lo es  movemos sin cambiar analagomente a como antes, y le agregamos  4 a r12d(lo movemos a la siguiente columna).\hfill \break
        Comparamos r12d con la cantidad de columnas (recuerden, las r12 = columna actual).Si es menor volvemos al loop, si no, vaciamos r12, incrementamos la fila (r11), los punteros de fila en la imagen de destino y fuente (rdi, rsi), y nos fijamos si r11 llego a la cantidad de filas pasadas(en ecx), si no es así terminamos, si no volvemos al loop. \hfill \break
        Por otro lado, si llego al borde_der, nos fijamos que r12d sea menor que r15d ( índice columna tenemos que dejar de procesar). Si no lo es, pasamos usando r13 los bits apuntados x r12 (*4 x píxel, + 8 porque estamos dos de desfajase)  le agregamos 4 a r12d (próxima columna) y volvemos a comparar si alcanzo el numero de columnas, y básicamente lo mismo descripto en el parrafo anterior, culminando en repetir el ciclo o terminarlo. \hfill \break
        En caso de ser menor, la dirección apuntada no forma parte de los bordes, (ya que para llegar hasta aquí tuvo que haber pasado todos las restricciones de bordes) y es una dirección a la que debemos aplicarle el filtro. \hfill \break
        Una cosa importante a aclarar en el funcionamiento del ciclo es que solo se pueden llegar a incrementa rdi, y rsi si se llega al limite superior o derecho, y que no pueden llegarse al limite superior e inferior a la vez (a menos que pasen una imagen 4x4 u menor, en la cual este filtro no haría nada), y lo mismo con izquierda y derecha,  por lo que si hay algo para procesar, luego de las primeras dos iteraciones iremos a procesar y no habremos movido el rdi, rsi de la posición inicial. \\
        Esto es importante porque estamos aplicando el filtro desfazadamente, de manera que si rdi apunta a (i,j) nosotros procesamos  {(i+2, j+2),(i+2,j+3),(i+2,j+4),(i+2,j+5)}. Si hubiéramos modificado rdi, o rsi, el desfajase haría que dejáramos lugares sin pasar. (r12 en cambio, apunta al primer píxel a procesar x iteración) 
        \hfill \break   
        \\
        Luego de aclarado esto, pasemos a lo que se hace una vez estamos en condición de hacerlo. Primero nos fijamos si r12d es 0 (osea, si empezamos fila nueva)
        \hfill \break
         
        Si lo es, leemos con rbx la dirección del contenido de rdi(fila actual) y lo movemos a xmm0.    
        \hfill \break
         xmm0 = $|a03|r03|g03|b03|a02|r02|g02|b02|a01|r01|g01|b01|a00|r00|g00|b00| $
         \hfill \break
 con rbx y direccionamiento,    movemos a xmm5 los 4 píxeles continuos (derecha) con movdqu también;
 luego incrementamos rbx en r8 (tamaño de fila de entrada, ahora rbx apunta a misma columna de la fila siguiente)
 movemos los 8 píxeles a xmm1 y xmm6, análogamente a como antes
 volvemos a incrementar la fila apuntada por rbx (de igual manera)
 y volvemos a mover los 8 píxeles a xmm2 y xmm7, respectivamente.
 \hfill \break
 En esta fila, iniciando en la parte alta de xmm8 y hasta la parte baja de xmm15,  están los 4 píxeles a procesar, por lo que para obtenerlos movemos a xmm8 y xmm15 los datos de los píxeles de xmm2 y xmm7
 luego usamos psrldq (ya lo hemos iso en sepia) en xmm8 con 8, para limpiar la parte alta del registro..de manera análoga, usamos pslldq para limpiar la baja de xmm15. Sumamos rxmm15 y xmm12 , de a bytes, y guardamos el registro en xmm15. 
 Luego, tenemos en r15 los píxeles efectivos a modificar. 
 \hfill \break
 Volvemos a aumentar de fila y mover los pixeles a xmm3 y xmm8, respectivamente, y una ultima vez mas, ahora guardándolos en xmm4 y xmm9.
 \hfill \break
 \\
En este momento, el contenido de los registros modificados:
\hfill \break
xmm4 = $|a43|r43|g43|b43|a42|r42|g42|b42|a41|r41|g41|b41|a40|r40|g40|b40|$
\hfill \break
xmm9 = $|a47|r47|g47|b47|a46|r46|g46|b46|a45|r45|g45|b45|a44|r44|g44|b44|$
\hfill \break
xmm3 = $|a33|r33|g33|b33|a32|r32|g32|b32|a31|r31|g31|b31|a30|r30|g30|b30|$
\hfill \break
xmm8 = $|a37|r37|g37|b37|a36|r36|g36|b36|a35|r35|g35|b35|a34|r34|g34|b34|$
\hfill \break
xmm2 = $|a23|r23|g23|b23|a22|r22|g22|b22|a21|r21|g21|b21|a20|r20|g20|b20|$
\hfill \break
xmm7 = $|a27|r27|g27|b27|a26|r26|g26|b26|a25|r25|g25|b25|a24|r24|g24|b24| $
\hfill \break
xmm1 =$ |a13|r13|g13|b13|a12|r12|g12|b12|a11|r11|g11|b11|a10|r10|g10|b10|$
\hfill \break
xmm6 = $|a17|r17|g17|b17|a16|r16|g16|b16|a15|r15|g15|b15|a14|r14|g14|b14|$
\hfill \break
xmm0 =$ |a03|r03|g03|b03|a02|r02|g02|b02|a01|r01|g01|b01|a00|r00|g00|b00|$ 
\hfill \break
xmm5 =$ |a07|r07|g07|b07|a06|r06|g06|b06|a05|r05|g05|b05|a04|r04|g04|b04|$
\hfill \break
 \\
 
y xmm15 =$|a25|r25|g25|b25|a24|r24|g24|b24|a23|r23|g23|b23|a22|r22|g22|b22|$
\hfill \break 
(los píxeles a procesar) (en el procesamiento no se modifican los registros de xmm5 a xmm9) \hfill \break
La notación usada separa cada xmm en 16 posiciones, en donde cada una anota el contenido de la siguiente manera... a,b,r u g representa el elemento de un píxel (r es rojo, b azul, g verde, y a el de alineación). A su vez, de los dos números que acompaña cada letra, el que es contiguo representa la fila del vecino en la que me encuentro (siendo 0, dos filas + abajo, 1 una fila mas abajo, 2 en la fila actual, 3 una fila mas arriba, y 4 2 filas mas arriba) y  el segundo, la columna vecina (de igual manera pero de 0 a 7, porque agarro ocho columnas por iteración (ya que proceso 4 píxeles en cada una))
\hfill \break
        Agregamos 8 a rbx para estar listo en la próxima iteración y vamos a modificar estos.\hfill \break
        
        Si no estamos en la primera columna de la fila, podemos aprovecharnos de los datos cargados en estos registros, donde, de los registros que no modificamos para aplicar el filtro, seguro hay alguno con contenido que volveremos a necesitar en la próxima iteración. El proceso de cargado es muy parecido, simplemente movemos la parte  xmm5 a xmm0, xmm6 a xmm1, xmm7 a xmm2, xmm8 a xmm3, xmm9 a xmm4, y en los anteriores, ubicamos el contenido de lo apuntado x rbx \hfill \break
        (rbx apunta a la fila rdi + la columna r12(*4 x píxel) + 8 (por que ahora ya tenemos la parte derecha de los vecinos cargada) a xmm5, sumamos fila y ahora movemos lo apuntado por rbx a xmm6, de nuevo a xmm7, volvemos a guardar el píxel a procesar en xmm15 (igual a como se hace al principio de la fila); se vuelve a sumar fila a rbx (sumando r8), se mueve contenido a xmm8, se vuelve a sumar a rbx y se pone el contenido en xmm9.\hfill \break
        Asi, tenemos de nuevo cargado los píxeles a procesar, utilizando la mitad de llamados a memoria. \hfill \break
        
        Una vez que tengo en los registros cargados, hacemos un puncpckbw (ya usado) entre xmm10 (contiene basura) y xmm0. y luego usando psrlw xmm10 (era el destino) y shifteando 8 cada word, limpiando la basura  y obteniendo los 2 píxeles menos significativos, con cada uno de sus elementos ahora en tamaño word.\hfill \break
        Solo para quedar mas claro
        \hfill \break
        
        \textbf{ Antes de punpcklbw (en tamaño byte)}
                \hfill \break
        xmm0 = $|a03|r03|g03|b03|a02|r02|g02|b02|a01|r01|g01|b01|a00|r00|g00|b00|$
                \hfill \break
        xmm10 = $|*|*|*|*|*|*|*|*|*|*|*|*|*|*|*|*|$
        \hfill \break
        \small{* significa basura}
        \hfill \break
\textbf{        luego de punpcklbw (en tamaño byte)}
\hfill \break
        xmm0 = igual
        \hfill \break
        xmm10 = $|a01|*|r01|*|g01|*|b01|*|a00|*|r00|*|g00|*|b00|*|$
        \hfill \break
        \textbf{ Luego  del psrlw (en tamaño byte)}
        \hfill \break
         xmm0 = igual
         \hfill \break
         xmm10 = $|0|a01|0|r01|0|g01|0|b01|0|a00|0|r00|0|g00|0|b00|$
        \hfill \break
        Hacemos lo mismo con la parte alta de xmm0, (con punpckhbw)ahora guardando en xmm11.
        \hfill \break
        Después, hacemos el mismo desempacamniento de xmm1, ahora la parte baja en xmm0 (que ya no necesitamos mantener) y en xmm12 la alta.\hfill \break
        Sumamos de a word (los elementos de los píxeles están separados de a word) xmm10 y xmm0; y también xmm11 y xmm12 (lo hacemos con paddw)
        Hacemos lo mismo con el contenido de xmm2, xmm3 y xmm4 (usando xmm0 y xmm12 como registros temporales.) \hfill \break
        \\
        Tenemos en xmm10 : $ |p41+p31+p21+p11+p01|p40+p30+p20+p10+p00|$\hfill \break
        Tenemos en xmm11 : $ |p43+p33+p23+p13+p03|p42+p32+p22+p12+p02|$
        
        
        (donde p, dado que i sea el numero significa ri +bi +gi) \hfill \break
        
        usamos psllq en ambos (con 8) para vaciar, de cada p, se valor a (rotamos los valores también)
        luego sumamos horizontalmente xmm10 y xmm11, y luego xmm10 con sigo mismo (como en sepia) para que nos quede (tanto en la parte alta como en la baja del registro ) las sumas que queremos. lo movemos a xmm0. \hfill \break
        Hacemos lo mismo (nada mas que empezando con xmm10, y ahora usando xmm1 y xmm2 como registros temporales) con las columnas de la derecha, de manera que al final queda:\hfill \break
        
 xmm0: $|p43+p33+p23+p13+p03|p42+p32+p22+p12+p02|p41+p31+p21+p11+p01|p40+p30+p20+p10+p00|p43+p33+p23+p13+p03|p42+p32+p22+p12+p02|p41+p31+p21+p11+p01|p40+p30+p20+p10+p00|$ (los ai de cada p en 0, y debido a las sumas horizontales, cada $pij =  |rij+gij+bij|)$
 \hfill \break \\
 xmm10 :$|p47+p37+p27+p17+p07|p46+p36+p26+p16+p06|p45+p35+p25+p15+p05|p44+p34+p24+p14+p04|p47+p37+p27+p17+p07|p46+p36+p26+p16+p06|p45+p35+p25+p15+p05|p44+p34+p24+p14+p04|$
 \hfill \break  \\
 utilizamos punpckhwd (que hace lo mismo que punpcklbw pero con words en vez de bytes y consiguiendo double words en vez de words) en xmm0 con sigo mismo para poder, limpiando la parte alta de cada doubleword (con psrld , y constante 8) tener en xmm0 todo lo que tenia anteriormente,  en doublewords (las partes repetidas las vacié en el shifteo) hago lo mismo con xmm10.
 \hfill \break
 Guardo xmm0 en xmm1. Luego, shifteo a la derecha 4 con psrldq para que quede ahora la doubleword menos significativa del registro sea la que antes esta en la 1ra posición del registro (viendo las posiciones como  de 0 a 3, de un doubleword cada una)y se la sumo a xmm0, el objetivo de esto es obtener cada  0 de xmm0  la suma de todos los sus vecinos contenidos en este registro, para lo cual es necesario repetir este procedimiento  2 veces mas.
 \hfill \break
 \\
Ahora, en xmm0:$|p43+p33+p23+p13+p03|p43+p33+p23+p13+p03+p42+p32+p22+p12+p02|p43+p33+p23+p13+p03+p42+p32+p22+p12+p02+p41+p31+p21+p11+p01|p43+p33+p23+p13+p03+p42+p32+p22+p12+p02+p41+p31+p21+p11+p01+p40+p30+p20+p10+p00|$
\\
\hfill \break
Ahora, sumamos a xmm0 con xmm10 ( todas la sumas en esta parte se hacen con paddd, para sumar de a doubleword). Al hacerlo, en la posición 0 de xmm0 ya tenemos todos los componentes de píxeles vecinos sumados del píxel menos significativo, resultado al que nos referiremos como sumargb(22).
\hfill \break
 Para obtener la suma correspondiente a los otros 3 píxeles que procesamos,  shifteamos a la izquierda xmm10 con pslldq y 4, (limpiando posición mas baja ocupada) y se lo sumamos a xmm0. Hacemos esto 2 veces mas.
 \hfill \break
Ahora, en xmm0:$ |sumargb25|sumargb24|sumargb23|sumargb22|$
\hfill \break
Por otro lado, utilizo pinsrd ;que copia una doubleword del operando fuente, (r10d tiene solo una) y lo copia en el operando destino; de acuerdo al tercer operando inmediato,   con xmm3, r10d(donde estaba el alpha) y 0. Osea que muevo el alpha  a la posición 0 de xmm3.
\hfill \break
Luego, utilizo pshufd en xmm3, consigo mismo y 0 para replicar el alpha en todas las doublewords del registro 
\hfill \break
(explicación: la operación copia doublewords del operando , y los inserta en las posiciones   según el tercer operando inmediato pasado. este inmediato es de ocho bits, donde lo que hay en los 2  bits menos significativos  indican cual posición del registro fuente es pasada a la posición 0 de la del destino, los siguientes 2 con la posición 1 y así con los 8. Como queremos la posición del fuente replicada en todas las posiciones del destino, alcanza con un 0 de 8 bits).
\hfill \break
Ahora, para seguir la formula, multiplico coda doubleword de xmm3 (las sumas) por alpha, usando pmulld (con xmm10 como destino), y lo copio a xmm1 y xmm2. xmm0, xmm1 y xmm2 tienen el mismo valor.
\hfill \break
Ahora movemos xmm15 (los píxeles principales) a xmm3. shifteamos de a doubleword a la izquierda 24, para quedarnos solo con los azules de cada píxel,  ( y luego 3 shifteamos a la derecha 24, para ponerlos en la parte baja de cada doubleword).  luego, copiamos xmm15 en xmm4 y shifteamos pero con 16 a izq, 24 a derecha, para quedarme con el verde. y lo mismo con el rojo y xmm10 (shifteando 8 a izq y 24 derecha).
\hfill \break
Terminamos  teniendo los azules de los píxeles en xmm3, los verdes en xmm4 y los rojos en xmm10, así que multiplicamos xmm0, xmm1 y xmmm2 (las sumas x el alpha) x ellos, y terminamos teniendo color ji*alpha*sumargbi 
(donde i va de 22 a 25, y color puede ser r, g o b), en cada uno.  luego los convertimos en floats de simple precisión para dividirlos por max_f (en xmm13), que tiene cuatro veces el máximo de la formula, usando divps.
\hfill \break
Ahora tenemos color ji*alpha*sumargbi/max, y nos falta sumarle a cada uno el color que le correspondía, lo hacemos sumándoles xmm3, xmm4 y xmm10 respectivamente. Conseguimos que en los registros destino estén la formula del filtro de cada color para los cuatro píxeles procesados. 
\hfill \break
Ahora solo nos queda reponer los a de cada píxel (y reordenar los colores, que cuando los vaciamos rotamos) Desempaquetamos los colores a words, packusdw contra sigo mismos. (nos quedan, los registros repetidos dos veces). Luego, utilizamos pshufhw, que es una versión de pshufd para words, y en la que voy, a explayarme. 
\hfill \break
Los pshfhw (y los pshflw) son hechos, siempre en este código con un registro y si mismo, para reordenar sus datos.
\hfill \break
\\
Ahora, el estado de xmm0, xmm1 es :
\hfill \break
xmm0: $|b25'|b24'|b23'|b22'|b25'|b24'|b23'|b22'|$
\hfill \break
xmm1: $|g25'|g24'|g23'|g22'|g25'|g24'|g23'|g22'|$\hfill \break
 colocamos g25' y los demás con apostrofe  para denotar que es distinto al color con el que empezamos.
\hfill \break
\\
queremos tener 
\hfill \break
 xmm0 : $|b25'|b25'|b24'|b24'|b23'|b23'|b22'|b22'|$
 \hfill \break
 xmm1 : $|g25'|g25'|g24'|g24'|g23'|g23'|g22'|g22'|$
 \hfill \break
usamos pshufhw y el inmediato 11111010b (y xmm0, luego haremos lo mismo con xmm1  ). Esto funciona así pshfhw nos mueve las palabras de la parte alta de un registro a la parte alta de otro, según las posiciones indicadas por el inmediato. En este caso, ahora viendo las posiciones de un registro como de 0 a 7 words, siendo 0 la mas baja,  manda la posición 6 de xmm0 a la posición 4 y 5, 
y la posición 7 a la posición 6 y 7. 
\hfill \break
xmm0:$|b25'|b25'|b24'|b24'|b25'|b24'|b23'|b22'|$
\hfill \break
luego usamos pshflw, que hace lo mismo en la parte baja, con inmediato 01010000b (y xmm0, y luego se hará con xmm1). Este manda la posición 0 a la 0 y uno, y la uno a la 2 y 3, por lo que queda 
\hfill \break
 xmm0 : $|b25'|b25'|b24'|b24'|b23'|b23'|b22'|b22'|$\hfill \break
 como queriamos. xmm1 es análogo. también lo es xmm12, (donde esta el rojo)
 \hfill \break \\
 Ahora que tenemos las cosas repetidas, shifteamos cada word, de xmm0 16 a la derecha, y a los de xmm1 a la izquierda, para que donde uno este en 0 el otro este el color buscado, y los sumamos de a word, obteniendo en xmm0 los verdes y azules en orden de los píxeles.
 \hfill \break
 Vuelvo a desempaquetar xmm0, ahora desempaquetado a bytes, (obtengo dos veces lo mismo repetido, no hay ninguna disturbancia porque, al aplicar la formula, lo que queda en cada color ES de tamaño de  byte(saturado)), luego, vuelvo a usar los mismos shuffle de antes (de a word), y me quedan como antes salvo que ahora, en vez de haber un color por word, esta el verde en la parte alta de cada word, y el azul en la baja. 
 \hfill \break
 Ahora, shifteamos a la derecha 16 bits cada dword para dejar la parte alta vacía, mientras conservamos los colores en la parte baja de estas. Shifteamos a la derecha 32 bits xmm2 para obtener el rojo en la posición que lo queremos. Volvemos a desempaquetarlo a  bytes, y lo shuffleamos como antes, y lo volvemos a shiftear 32 bits luego de eso, pero a la izquierda, para que quede en la pocision correcta para sumar con xmm0.
 \hfill \break
 xmm0 = $|0|r25'|g25'|b25'|0|r24'|g24'|b24'|0|r23'|g23'|b23'|0|r22'|g22'|b22'| $
 \hfill \break
  Sumamos los dos de a byte. Y ahora solo nos queda poner los a, lo cual hacemos agarrando xmm15,(píxeles a procesar sin tocar) y los shifteamos a la izquierda cada doubleword 24 bits, (para vaciar todo salvo las a de cada píxel) y luego 24  bits a la derecha para que queden en la parte mas significativa de cada word; luego suma de a byte con xmm0 como destino y deja en este los cuatro píxeles con el filtro aplicado.
  \hfill \break
  Movemos el puntero de píxeles a rsi (puntero a imagen de salida), le sumamos 2 filas a rbx y movemos xmm0 al contenido de rbx + columna actual (r12 ,*4 x los píxeles) + 4 x el desfajase. 
  \hfill \break
  Incrementamos el puntero a  columna actual (r12d), y preguntamos si acabo la fila, si no lo hizo iteramos de nuevo, si lo hizo, lo reiniciamos, (xor) incrementamos el contador de filas, y  los punteros de las imágenes fuente y destino para que avancen una fila (sumándoles tamano de fila), y si r11 es menor a rcx (cant de filas) entonces iteramos de nuevo, si no, terminamos la imagen, por lo cual desarmamos la pila y salimos. 
 \hfill \break
 

 







 
 

  
 

    
   
 



\section{Experimentación}
\hfill \break
Para cada filtro y en cada experimento usamos dos muestras de imágenes. Una muestra es generada al azar, o sea el valor de cada pixel de la imágen es determinado de forma azarosa, mientras que en la otra todos sus píxeles tienen el mismo valor, típicamente negro (00000000h) salvo que se indique lo contrario. Usamos dos muestras para mostrar dos casos extremos, uno en que los pixeles estan fuertemente relacionados (la imagen constante) y otro en el que no (la imagen aleatoria), y poder ver si hay presente alguna optimización por parte del procesador en alguno de los casos. 
\hfill \break
Como la variación es muy grande entre mediciones, decidimos realizar 100 de estas por cada muestra. Una vez obtenidos los datos nos quedamos con las 10 más chicas y calculamos el promedio. No podemos asegurar que todos los outliers sean removidos de esta forma, pero consideramos que es un método lo suficientemente estable ya que podamos bastantes valores (el 90\%). 
\hfill \break
La métrica que usaremos en todos los casos para comparar la performance de las implementaciones sera ticks por pixel. Consideramos que es una forma justa de comparar dos implementaciones ya que como usan una cantidad de memoria constante (las de Assembler usan solo registros y las de C usan una cantidad fija de variables), lo único que diferencia la performance de ambas es el tiempo de ejecución. Por lo tanto si una implementación requiere menos ticks en la gran mayoria de los casos, es entonces una implementación cuya performance es superior. Si se nos escapa la palabra rapido, o menos tiempo, nos estamos refiriendo a esto, tambien. 
\hfill \break
\subsection{Cropflip}
\hfill \break
Para este filtro realizamos dos experimentos. El primero consta en medir muestras variando el ancho y alto de estas. De aqui podremos observar como influyen ambos parámetros. El segundo experimento consta en tomar muestras cuadradas e ir incrementando el tamaño. Con este experimento queremos determinar si hay alguna caida de performance al incrementar el volumen de los datos. En ambos casos fijamos los parametros para que no haya recorte, ya que recortar es equivalente a aplicarle el filtro una imágen de tamaño más chico. (si usamos recorte en alguna descripcion nos estaremos refireindo a imagen de menor tamano)
\hfill \break
Hipotesis: Nuestras hipotesis al primer experimento son que al aumentar el ancho y alto de las muestras tomadas, la cantidad de tiempo de las mediciones sera mayor, y, como la implementacion de assmebler procesa de a 4 pixeles y la de c de a uno x ciclo,  el assembler sera 4 veces mas rapido.
En cuanto al segundo experimento, creemos que incrementar el volumen de los datos a procesar aumente el tiempo de procesamiento, pero deberia mantener la diferencia postulada anteriormente. 
Una hipotesis adicional en ambos casos es que, como al shiftear en cropflip no se toma en cuenta el color e la imagen, no tendria que haber mucha diferencia entre las imagenes constantes y las generadas aleatoriamente

\hfill \break
 Los resultados los mostramos en los siguientes gráficos:
 

\hfill \break


En el primer experimento, podemos notar la diferencia de performance entre el c y assembler viendo las barras de color, que mientras van de 1 a 2,75 en assembler, van de  22.16 a 21.68 es c, varias veces mas (de hecho, mas de cuatro). En el aleatoria assembler, vemos como va aumentando los ticks por pixel a medida que la imagen se hace mas grande. En el aleatorio c, mientras que al principio (por cache sin datos) tarda mas que nunca, luego su diferencia de performance al aumentar la magen es menor a la del assembler, (por cache y por que es un proceso mas uniforme de acceso con una esrtuctura definida (matriz) con un tiempo de acceso, en vez de el recorrido por punteros en assembler. De todas maneras, el assembler es mas efectivo


\hfill \break


Aqui estan las constantes de assembler y c con este mismo experimento. Al assembler no hay mucha diferencia, pero en el c la hay, tanto que la optimizacion de ticks del aleatorio no es aprovechada.
\hfill \break
con respecto al segundo experimento

\hfill \break
Como podemos ver en el grafico, estan los ticks por pixel de las corridas de assembler y c, dados imagenes cuadradas, donde vamos aumentando su tamano, y realizamos siempre el mismo recorte. 
Podemos notar que al aumentar el tamano de imagen, los ticks por pixel de todas las implementaciones c aumentan, asi que esa parte de la teoria es correcta. En las implementaciones assembler se da la pecularidad de que el 100 y 1000 tardan menos que el de 10, en parte se debe a la cache, (una implemntacion que es un poco mas grande a 10 aprovecha la cache mas, tal vez 10 es demasiado chico), podemos ver que de 100 en adelante, se cumple para el aleatorio.  
	En cuanto a la relacion de tiempos entre el c y el assembler, la menor diferencia que hubo fue entre la implementacion de la imagen de 10000x10000, y es un poco menos de 4 veces el tiempo del assembler, pero igualmente una diferencia de 3 veces por lo menos. Como hemos notado antes, no hay bastante diferencia  entre la imagen constante y la no constante en el c, pero nos sorprende la diferencia que se aprecia de este tipo en el assembler en las imagenes.   
	 
 \hfill \break
En conclusion, tanto el usar imagenes mas grandes, como el pedir dos recortes, uno mas grande que otro sobre dos imagenes de igual tamano, hara que los tics de pixeles aumenten. Un recorte de determinado tamano tardara mas en una imagen constante que una aleatoria, pero si aumentamos el tamano de  recorte la diferencia deisminuira. 

\hfill \break


\subsection{Sepia}

Para este filtro vamos a realizar dos experimentos. Son los mismos dos experimentos realizados para cropflip, pero con distintas hipotesis.
\hfill \break
 
   Vamos a decir que el juego de los tamanos de agarrar imagenes cuadradas tenia mas sentido en el cropflip, por lo que solo se hara el analisis con las distintas imagenes constantes y de tamano. 
   \hfill \break
   Hipotesis:  NUestras hipotesis son que el tamano de la imagen hace que tarde mas, y que la imagenes constantes son mas rapidas de procesar que las aleatorias.
   
   
   
   
   \hfill \break
   
   
   Podemos notar que  el aumento de ticks por pixel en el c de la imagen constante es proporcional a su tamano uniformemente (con este digo, la variacion es constante y proporcionada al incremento), no solo eso, es mas rapido que el assembler, ( o tal vez no, porque aunque la escala de color de assembler es mayor  hay mas cuadrados con menor escala); pero es alucinadamente rapido.eso se debe a tener que replicar el mismo resultado sin tener que hacre cuentas, por acceder siempre a iguales datots, ya el procesador pone el resultado; o por estar en el periodo alto de la curva de la cache, si la entrada fuera mas grande, la cach se empezaria a vaciar y el rendimiento caeria.
   
   
   
   
   \hfill \break

 Ahora vemos el comportamiento de las implementaciones con las imagenes aleatorias. La diferencia de tiempos en el caso de c comparado a la  uniforme es muy significativa. (al igual que con la del assembler) Notar 
que por el contrario, a la implementacion assembler es mas eficiente de esta manera,  por que, como son imagenes de puro negro, las sumas en assembler generalmente tienen que ser saturadas, o lidiar con los carries, o por las multiplicaciones de punto flotante de numeros mas grandes.





\hfill \break


Conclusiones: El programa c performa mejor si el tamano de imagen es determinado y constante que el assembler, pero en cualquier otro caso el assembler es mucho mas rapido. Las imagenes aleatorias son mas faciles para el assembler que las constantes (negra). 

\hfill \break







\subsection{LDR}
\hfill \break
Para este filtro vamos a realizar un experimento. Es el primer  experimento realizado para cropflip, pero con distintas hipotesis.

\hfill \break
Hipotesis: En este filtro cambiamos paleta, al igual que en sepia, pero el cambio de paleta no es independiente de cada pixel, si no que es afectado por los pixeles rondeantes. Tomando esto en cuenta, nos parece logico pensar que si la imagen es constante, el rendimiento se vera afectado solo por las dimensiones de la imagen, y en el aleatoria el tiempo variara tanto por las dimensiones como por el contenido, pudiendo ser en casos menor a la de la constante. El ldr en asm implementado ahorra muchos usos a memoria por rehusar datos cargados en el ciclo anterior, optimizacion que el asssembler no posee, por lo que el assembler entra la mitad de veces a memoria por fila, lo que, a mayor cantidad de filas, hace que sea menor; en cualquier caso, sera siempre mejor 8 veces (4 por cantidad de pixeles por vez. 4 x memoria).    


\hfill \break




Aqui vemos el c y assembler con la figura constante, como podemos nota, el c denuevo de una distribucion bastante uniforme , aunque, comparado con el sepia, en esta ocacion aumentar el tamano aumenta mas la complejidad, si la otra aumentaba por una constante, esta lo hce de manera cuadratica) eso se debe a que la formula realizada requiere pixeles vecinos. Otra diferencia con el sepia es que esta vez los tiempos no fueron magicamnete menores a los del assembler, por, yo creo, lo mismo de antes. 
El assembler muestra una performance 8 veces mas rapida (de hecho, 16 veces + rapida) y el tiempo del filtro parece casi el mismo sin importar el tamno. Esto se debe a que no elegimos un tamno de imagen lo suficientemente alto, pero que al aumentar como lo hicimos la imagen no se viera afectado en si la rapidez demustra que se optimiza mucho aprovechando los vecinos de veces anteriores.
\hfill \break


Aqui vemos el c y el assembler con la figuras aleatorias, aqui el assembler es incluso mas rapido en la mayoria de los casos, creemos que porque el contenido afecta lo las imagenes, asiendolas mas faciles de usar en cuentas y procesar. En cuanto al c,   las imagenes con pocas columnas o ocas filas son mas rapido, y la discrepancia es menor entre las imagenes grandes que las constante. Esto se debe a que saturamos menos (en c se hace con ifs en vez de instrucciones, que tarda mas) que con una imagen totalmente en negro. 
\hfill \break




Conclusiones: El tamano de la imagen no es un factor tan grande como pensabamos en este filtro las imagenes aleatorias grandes se procesan mas rapido en  el c que las constantes, pero si se reviertenlos tamnos, tenemos algo parecido. El assembler funciona rapido siempre, pero mas si es aleatoria, y aunque la diferencia es poca, tambien si hay ms filas que columnas si la imagen es mas grande, esto pierde distancia, pero en las mas pequenas se nota).



















 
 
    	
	  
	
	 
	 
	
	
	   
		
	
	
	.
	
	
	
	
	 

	
	
	  



\section{Conclusiones y trabajo futuro}



\end{document}

